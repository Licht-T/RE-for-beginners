\mysection{\HelloWorldSectionName}
\label{sec:helloworld}

[\KRBook]という本の有名な例を使ってみましょう

\lstinputlisting[caption=\CCpp Code,style=customc]{patterns/01_helloworld/hw.c}

\subsection{x86}

\input{patterns/01_helloworld/MSVC_x86}
\input{patterns/01_helloworld/GCC_x86}
\input{patterns/01_helloworld/string_patching_JPN}

\subsection{x86-64}
\input{patterns/01_helloworld/MSVC_x64}
\input{patterns/01_helloworld/GCC_x64}
\input{patterns/01_helloworld/address_patching_JPN}

\input{patterns/01_helloworld/GCC_one_more}
\input{patterns/01_helloworld/ARM/main}
\input{patterns/01_helloworld/MIPS/main}

\subsection{\Conclusion{}}

x86/ARMとx64/ARM64コードの主な違いは、文字列へのポインタ長が64ビットになったことです。
現代的なアプリケーションによる広大なメモリ領域の要求と,メモリ価格の下落に後押しされ,現代的な \ac{CPU} は64ビット化されました。
それによって,32ビットポインタを使用していたときよりも,はるかに多くのメモリをコンピュータ上で利用可能となったのです。

% sections
\input{patterns/01_helloworld/exercises}
